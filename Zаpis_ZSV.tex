\documentclass[10pt,a4paper,
twoside,%twocolumn
%,landscape,twocolumn
]{report}
\usepackage[utf8]{inputenc}
\usepackage[czech]{babel}
\usepackage[ampersand]{easylist}
\usepackage{graphicx,lipsum}
\usepackage{framed}
\usepackage[pagestyles]{titlesec}
\titleformat{\chapter}[block]{\bfseries\Huge}{\thechapter}{15pt}{\vspace{-20pt}}[\vspace{-0.5ex}]
\usepackage{wrapfig}
\usepackage[T1]{fontenc}
\usepackage{amsmath}
\usepackage{multicol}
\setcounter{secnumdepth}{2}
\setcounter{tocdepth}{2}
\usepackage{graphicx}
\usepackage{multicol}
\usepackage{subcaption}
\usepackage{layout}
\usepackage{amsfonts}
\usepackage{amssymb}
\usepackage{graphicx}
\usepackage{hyperref}
\usepackage{fancyhdr}
\usepackage[dvipsnames]{xcolor}
\lfoot[Jakub Dokulil]{Jakub Dokulil}
\rfoot{}
\cfoot{\thepage}
\author{Jakub Dokulil}
\title{Základy Společenských Věd}
\usepackage{marginnote}
\usepackage[top=2cm, bottom=2cm, outer=1.25cm, inner=1.75cm]{geometry}
\makeindex
\pagestyle{fancy}
\linespread{1.2}

\newcommand{\ust}[1]{\begin{footnotesize}%
\begin{center}
Článek #1
\end{center}
\end{footnotesize}}%

\begin{document}%%%%%%%%%%%%%%%%%%%%%%%%%%%%%%%%%%%%%%%%%%%%
\pagenumbering{Roman}
\begin{titlepage}
\maketitle
\end{titlepage}

\vspace{10cm}
\pagestyle{empty}
Zápisky z hodin ZSV. Vysázeno pomocí systému \LaTeX.

\newpage
\pagestyle{plain}

\begin{multicols}{2}
\tableofcontents
\end{multicols}
\pagenumbering{arabic}
\part{Sociologie a Politologie}
\chapter{Sociologie}
\pagestyle{fancy}
\section{Moc}
Autorem Teorie moci je\index{Max Weber} \textbf{ Max Weber}. Podle něj je \index{moc} moc schopnost prosadit názor, zájem, cíl i přes nesouhlas druhých. \\
Moc člověk může dosáhnout

	1. Násilí\\
		a. Fyzické\\
		b. Psychické\\
		c. Ekonomické
\begin{enumerate}
	\item Tradiční moc\\
	• Moc se předává z generace na generaci (Např. monarchie)
	\item Charismatická\\
	• Člověk má moc díky své osobnosti, hýbající celosvětovými masami\ldots(Papež, Dalajláma)
	\item Legální\\
	Základní rozdíl je v dodržování lidských práv.
	\begin{enumerate}
	\item Diktatura \index{diktatura} \begin{itemize}
		\item Moc je stejně rozdělená jako u demokracie, akorát je ovládána jednou osobou či stranou
	\end{itemize}
		 
	\item Demokracie \index{demokracie} \begin{itemize}
		\item Dodržují se lidská práva, existuje občanská společnost
		\item Existuje občanská společnost, tj. jiné společnosti než pol. strany
	\end{itemize}
	\end{enumerate}
\end{enumerate}
\section{Sociální struktura}
\begin{quote}
\emph{Síť vztahů mezi jednotlivci.}
\end{quote}
Tato síť zahrnuje také Spol. role, soc. pozice\ldots
\subsection*{soc. postavení} \index{postavení ! sociální}
Závisí na \begin{itemize}
\item Vzělání
\item majetek (příjem)
\item spol. prestiž (moc)
\end{itemize}
\subsection*{Nerovnost} \index{nerovnost!ve spol. postavení}
Důležitý prvek rozvrstvení ve společnosti, \textbf{řada neovlivnitelných okolností kvůli kterým si lidé nomohou být spol. rovni}, souvisí s rozdíly  v majetku, rase, pohlaví, vzdělání\ldots 
\subsection{Sociální mobilita}
\textbf{Sociální mobilita} znamená možnost pohybu ve soc. struktuře mezi jednotlivými vrstvami.
\paragraph{Vertikální Soc. mobilita}
Posun nahoru je např.: nabytí majetku, povýšení, dokončení vzdělání, karierní růst
Posun dolů např.: být dlouhodobě nezaměstnaný, alkoholismus, závislost, finanční krize, osobní krize, 
\paragraph{horizontální soc. mobilita}
Nemění se status, ale pouze místo kde člověk žije
\paragraph{Prostorová mobilita} \index{mobilita!prostorová}
Rozdíl v místě kde člověk bydlí.
\paragraph{Mobilita v závislosti na rozdílu věku}
\subparagraph{intergenerační mobilita} se zajímá o výhodnost podmínek pro různé generace.
\subparagraph{intragenerační mobilita} je porovnání podmínek  lidí stejně starých v různých obdobích.

\chapter{Politologie}
\section{Úvod}
Politologie vzniká ke konci 19. století oddělením od f\mbox{}ilozofie. \emph{Polis-logos} (řecký městský stát-politika (zájem o veřejné věci)-věda).

Politologie se zabývá o fungování státu, jeho institucí, státní správy, činnost politických stran, jejich programy, politické chování lidí, spolupráci s cizími státy.

Stát musí mít území, obyvatele, státní správu a instituce, suverenitu vůči ostatním státu a státní symboly.
Stát plní fce kulturní, ekonomické, právní, bezpečnostní, sociální\ldots

Po vzniku politologie jsou dva přístupy:
\begin{description}
\item[Americká politologie] je pragmatická, hodně analytická, využívá metody statistiky a modelování
\item[Evropská politologie] konzervativní přístup, tendence, vyhodnocovat a dávat návody k řešení, \emph{filozofuje}
\end{description}
\paragraph{Počátky:} Politologie se zpočátku zabývá důvody vzniku státu. Později se vznikem polit stran se hodně zabývat o participaci (začlenění do moci)

\paragraph{Dnešní politologie} Politologové působí poradci státníků. 
\subsection{Vztah politologie a ostatních spol. věd}
\subsubsection{Vztah k filozofii}
Filozofové mají zájem o ideální uspořádání společnosti. (Platón, Aristotelés, sv. Augustýn, Machiavelli)
\subsubsection{Vztah k historii}
Je dán souvislostí historických událostí a politických rozhodnutí. 
\subsubsection{Vztah k sociologii}
Souvislost společnosti a jejího řízení.
\subsubsection{Vztah k ekonomii}
Politici řídí ekonomiku. Politická rozhodnutí se projevují v ekonomické situaci.
\subsubsection{Vztah k psychologii}
Zastrašování mocnými. Využití volebního práva.
\subsubsection{Vztah k právu}
Znalost právních a volebních systémů.

\begin{figure}
beginfig(1);
draw (0,0)--(10,0)--(10,10)--(0,10)--(0,0);
endfig;
end;
\end{figure}

\section[Představitelé]{Nejvýznamější představitelé polit. myšlení}

\subsection{Sofisté}
\emph{5 stol. před našim letopočtem}

\uv{Sekta} lidí, představtel \textsc{Prothagoras} přichází s myšlenkou že společnost nepotřebuje pravidla, ale stačí se řídit \textbf{právem silnějšího} (těm je dovoleno vše).

\subsection[Platón]{Platón (427 -- 347)}

Dílo \emph{\textbf{Politea}} - ústava. Skládá se ze dvou částí, v těch:
\begin{description}
\item [Kritizuje] všechny tehdejší formy vlády (oligarchie, demokracie - každý by si mohl dělat co chce $\rightarrow$ anarchie popř. tyranie)
\item [Dává vlastní představu státu] rozdělení společnosti do tří stavů, každý se může dostat do každého stavu\begin{itemize}
\item vládců (vzdělaní, \textbf{moudří} lidé)
\item strážci (odvážní lidé)
\item výrobci (umírnění lidé)
\end{itemize}
\end{description}

\subsection[Aristotelés]{Aristotelés (384 -- 322)}

\paragraph{Dílo politika} V tomto díle tvrdí že člověk je \emph{zoon politikon} (tvor společenský), tzn. že k životu potřebuje společnost lidí, lidé zakládají rodiny, z těch jsou obce z těch je stát, ten potřebuje \textbf{vládu, ústavu, obranu}.


\subsection{Aurelius Augustin (354 -- 430)}
\paragraph{O obci boží} Augustin tvrdí, že existují dva světy:
\begin{itemize}
\item \textbf{Pozemský svět} nedokonalý, plný hříchů
\item \textbf{Svět božího království} dokonalý svět, 
\end{itemize}
Augustýn tvrdí že lidé jsou zatížené už při narození hříchem, budou li se lidé řídit radami církve pak se dostane do "dokonalého světa".

\subsection[N. Machiavelli]{Niccolo Machiavelli (1469 -- 1527)}

\paragraph{Vladař} V díle je návod jak se chovat jako panovník, tvrdí že zásadní pro každý stát být sjednocený a silný. Tvrdí že chcou-li lidé se dostat k moci tak stačí pokud zlikviduje svého předchůdce a jeho rodinu a sympatizanty v co nejkratší době. Tvrdí že lidem je jedno kdo vládne, jen se jim nesmí sahat na majetek. 

\subsection{Osvícenství}
\subsubsection{J. Locke}
\textbf{Dvě ojednání o vládě} v tomto díle dělí státní moc na zákonodárnou a výkonnou. Tvrdí že panovník musí lidem zajistit jejich práva, zejm. na soukromý majetek, pokud tak neudělá lidé maí právo ho odvolat.
\subsubsection{Charles Montesuieu}
 V díle \textbf{Duch zákonů} navazuje na Locka přidáním nezávislé soudní moci.
 
 \subsubsection{J. J. Russeau} \label{russeau}
Společenská smlouva 
 idea o ideálním státu. Na jedné straně se lid dohodne s panovníkem a vytvoří ústní smlouvu. Lid se zavazuje že se nechá svázat zákony a pokud je poruší pak přijme trest. Panovník se zavazuje k respektování zákonů a řízejńí se jimi.

\subsection{Čeští předstvitelé politologie}

\subsubsection{Česká politika do 19. století}
\paragraph{Jan Hus} Kritizuje 

\section{Politika}
Politika z hlediska politologie označuje rozhodování určité skupiny lidí či jedince se zájmy řízení a vedení státu či společnosti. V tomto může být vnímána jako \textbf{koncepce} (program) nebo jako místo kde se programy střetávají (PS, Senát, zastupitelstvo\dots).

Politický program lze rozdělit na zásadní a dílčí.

\paragraph{Zásadní politický program} Takový program vychází z ideologie strany (např. levicová strana má zvýšení důchodů\dots)

\paragraph{Dílčí polit program} Tím jsou kroky kterými lze zásadní program naplnit.

\subsection{Politický pluralizmus}
Podstatou tohoto pluralizmu je že ve společnosti existuje široké spektrum polit stran, které se snaží dostat volbami k moci (participovat se na ní) a prosadit program.
\subsection{Politickék programy, Pravice $\times$ Levice}
Poprvé se rozdělení Pravice, Levice po Velké Fr. revoluci. Název odpovídá tomu kdo seděl na jaké straně od panovníka. Na pravici seděli ti kteří kladli důraz na silnou vůli panovníka a tradic, zdůrazňovali lid. práva. Po levici seděli příznivci omezení moci panovníka, zdůrazňovali rovnost a solidaritu.
\paragraph{Pravicové strany:} na prvním místě je člověk jako jedinec (aby sám aktivně přebral odpovědnost), lidská práva a svobody, podpora podnikání, principy demokracie  (TOP09, ODS)

\paragraph{Levice:} bere člověka jakou součást kolektivu, stát má kontrolu nad všemi oblastmi života, \uv{svazuje lidi} (ČSSD, KSČM)

Středové strany: KDU-ČSL, ANO,

Exttrémisté: SPD, Pirátská Strana, 

\paragraph{Antipolitický program:}
Zájmy mezinárodního míru, protiválečné zájmy, lidská práva 

\paragraph{Negativistický prog.} Skupina potřebuje voliče aby se dostala k moci a pak absolutisticky vládla.

\paragraph{Pragmatický program:} Je vycházející z aktuálních potřeb voličů, lze říct že je lehce populistický (zvýšení příjmů, snížení nezaměstnanosti\dots )

\paragraph{fundamentalistický program:} Prosazování nějaké ideologie či náboženství za  každou cenu.

\section{Stát}

Stát je politická organizace společnosti.

\subsection{Teorie vzniku státu}
\paragraph{Mocenská t.} Machiavelli mluví ve Vladařovi o násilném přisvojení území či získání moci.

\paragraph{Smluvní teorie} viz. Spol smlouva (Russau), viz \ref{russeau}.

\paragraph{Náb. teorie} Mluví o zásahu boží vůle.

\paragraph{Konstitutivně akceptační teorie} Stát vzníká na základě rozhodnutí světových velmocí (např. Izrael)

\subsection{Znaky státu}

\paragraph{Území} Je ohraničený suchozemský prostor s pobřežními vodami vč. vzdušného a podpovrchového prostoru.

\paragraph{Občané} Lidé žijící na území státu a spadající pod státní správu.

\paragraph{St. správa} ta je hierarchivky uspořádána (PS, zastupitelstva...)

\paragraph{St. symboly} Znaky, vlajka\dots

\paragraph{Suverenita}
\subparagraph{Vnější suverenita} dána vztahem s vnějšími státy.
\subparagraph{Vnitřní suverenita} stát garantuje lidem práva, vztah občan -- stát (je dán právem)

\subsection{Funkce státu}
\begin{description}
\item[Právní fce] spočívá v tom že vztah občan-stát je dán právem\footnote{Stát může pouze to co zákon dovoluje. Občan může vše co zákon nezakazuje.}
\item[Ekonomická fce] její podstatou je že stát musí vytvořit podmínky pro chod ekonomiky.\footnote{také zajistit možnost podnikání a trž. hospodářství.}
\item[Sociální fce] -- Stát by se mě postarat o staré, nemocné a lidi kteří jsou bez práce.
\item[Bezpečnostní fce] -- stát by měl garantovat ochranu zdraví a majetku občanů.

\item[Kulturní fce]-- podpora tradic památek a zajíišťování vzdělávání a vědy.
\end{description}
 
\paragraph{Vnější fce:} Diplomacie podpora zahr. obchodu, bezpočnostní fce.

\subsection{Dělení států}
\subsubsection{Dělení podle suverénů (kolik lidí vldne)}

\begin{description}
\item[Monarchie] vládne jeden

\item[aristokracie (oligarchie)] vldne skupina
\item[Demokracie] vládne lid
\end{description}

\paragraph{Podle \uv{hlavy} státu}
\begin{description}
\item[Republiky] hlava státu je volena, ty pak mohou být podle dělení:
\begin{description}
\item[Prezidentské] V takové republice prezident reprezentuje výkonou moc, avšak se musí řídit zákony a nesmí rozpouštět zákonodárný sbor. (USA, Argentina, Jižní Korea \dots)
\item[Parlamentní] Prezident v tomto případě m hlavně reprezentativní moc a jeho pravomoce nejsou silné. Vláda se pak zodpovídá parlamentu. (ČR, SR\dots)
\item[Semiprezidentská] V takovém případě prezident má silné postavení avšak premiér má rozhodující slovo. (Francie, Rusko)
\item[Kancléřská republika]Existuje prezident avšak kancléř má výraznější postavení jak premiér v parlamentní rep. (Rakousko, Německo)
\end{description}

\item[Monarchie] moc se předává z generace na generaci
\begin{description}
\item[Absolutistická] Monarcha stojí nad zákonem, může co chce disponuje armádou \dots (Saudská arábie, katar, Omán\dots)
\item[konstituční] V takové monarchii je moc panovníka omezana ústavou. Panovník se musí řídit zákony a dělit se o moc. (Monako, Maroko, Kuvajt, Japonsko\dots)
\item[Parlamentní] Panovník takřka nemá pravomoce, poze reprezentuje moc (Spojené Království, Švédsko, Španělsko\dots)
\end{description}
\end{description}

\subsubsection{Státy podle zřízení}
\begin{description}
\item[Demokracie]Moc vychází z lidu
	\begin{itemize}
	\item dodržování lid práv
	\item moc vycházející z lidu
	\end{itemize}

\item[Diktatura] vládne jeden (Bělorusko, Sev. Koerea, Čína, Venezuela, Kuba)
	\begin{itemize}
	\item struktura a dělení moci podobná jako u demokracie
	\item moc je ovládána Jednou osobou nebo stranou
	\end{itemize}
\end{description}

\subsubsection{Podle územní celistvosti}
\begin{description}
\item[Federace] -- USA, Německo, Rakousko
	\begin{itemize}
	\item Stát je složený z několika menších samostatných územních celků, ty mohou mít vlastní legislativu.
	\item Všechny státy mají společnou ústavu, hlavu státu, měnu, obranu \textbf{zahraniční politiku}.
	\end{itemize}

\item[Konfederace] -- dnes takřka neexistuje, podobné znaky nese EU
	\begin{itemize}
	\item Seskupení více států.
	\item Legislativa členských zemí musí být v souladu s tou nadřazenou.
	\item V rámci zahraniční politiky státy mohou vystupovat samostatně.
	\end{itemize}

\item[Unitární stát] -- stát nečleněný na samostatné územní celky (ČR)
\end{description}

\subsubsection{Soustředění moci}
\begin{description}
\item[Centralizované státy] moc je řízna z jednoho centra. (Německo, Rakousko)

\item[Decentralizované] moc je řízena z více center. (ČR, JAR)

\end{description}


\section{Státní symboly ČR}

\paragraph{Velký státní znak}
čtvrcený štít v 1. a 4. poli je Český 2 ocasý lev ve skoku, 2. pole obsahuje Moravskou orlici, ve 3. poli černá slezská orlice se stříbrným půlměsícem na hruudi. Zvířata se dívají do prava.

Takový znak musí být vyvěšen na státních institucích

\paragraph{Malý státní znak} Stříbrný český dvouocasý lev ve skoku. Používá se např při úředním jednání.

\paragraph{Státní vlajka} Tvar obdélníku (strany v poměru 2:3), do poloviny delší strany zasahuje modrý klín, nahoře je bílá, dole je červená. Modrý klíd byl vložen až roku 1920 s únorovou ústavou.

\paragraph{Státní barvy} Nejčastě v podobě trokolory. Bílá, červená a modrá jsou státními barvami.

\paragraph{Standarta prezidenta republiky} Vlajka prezidenta republiky je bílá, s okrajem skládajícím se z plaménků střídavě bílých, červených a modrých. Uprostřed bílého pole je velký státní znak. Pod ním je bílý (stříbrný) nápis PRAVDA VÍTĚZÍ na červené stuze podložené žlutými (zlatými) lipovými ratolestmi.

Viz příloha 4 zákona 3/1993 sb, \ref{fig:standarta}.

\paragraph{Státní pečeť} viz příloha 5 zákona 3/1993 sb.

\paragraph{Státní hymna} Státní hymnu tvoří první sloka písně Františka Škroupa a Josefa Kajetána Tyla \uv{Kde domov můj}.

\paragraph{Insignie} Svatováclavská koruna, žezlo a jablko.

\begin{figure}\centering
\begin{subfigure}{0.38\textwidth}
\includegraphics[width=\textwidth]{standarta}
\caption{Vlajka prezidenta republiky}
\label{fig:standarta}
\end{subfigure}~\hspace{0.5cm}\begin{subfigure}{0.38\textwidth}
\includegraphics[width=\textwidth]{pecet}
\caption{Státní pečeť}
\label{fig:pečeť}
\end{subfigure}
\caption{některé státní symboly.}
\end{figure}

%\printindex

\part{Právo}
\chapter{Ústava}
\begin{quote}
\emph{Ústava je základní zákon státu, který má nejvyžší právnín sílu.}
\end{quote}

\begin{description}
\item[Původní] originální ústava vytvořená tou danou zemí
\item[Přenesená] je ústava ovlivněná ústavou jiného státu
\end{description}

dále jsou ústavy psané a nepsané
\begin{description}
\item[Psaná ústava] od prvopočátku má napsanou podobu, je to dokument se spec uzákoněním
\item[nepsaná] vzniká na základě tradic a vzniků
\end{description}

Ústava je \textbf{flexibilní} pokud ji lz snadno měnit.

\textbf{Rigidní ústava} je ústava kterou je obtížné změnit.

ústava musí obsahovat
\begin{itemize}
\item princip dělení státní moci
\item princip brzd a rovnováh, např. vláda potřebuje důvěru sněmovny na to vše dohlíží nezávislá soudní moc
\item způsob jmenování a volby představitelů moci
\end{itemize}
\section[Ústavní vývoj]{Ústavní vývoj na území ČR}
\paragraph{Vznik ústavy} Ústava může být:
\begin{description}
\item[revoluční] je výsledkem revoluce
\item[dohodnutá] schválena zákonodárci
\item[oktrojovaná] je vydána panovníkem bez suhlasu parlamentu
\end{description}

Ústava prochází 3 obdobími. Často ke změnám Ústavy dochází na základě změn spol. poměrů. Jejich názvy jsou často odvozeny od měsíce ve kterém byla schválena.

\subsection{1. období (1848--1918)}
\paragraph{Prosincová ústava 1867} Ačkoli jsme součástí Rak-Uhr. tak je pro nás význam, nembyla oktrojovaná. Tato ústava má \textbf{6 hlav}:
\begin{easylist}[enumerate]
\ListProperties(Style2*=,Numbers=a,Numbers1=R,FinalMark={.})
& Uzákonění Říšské rady jakožto zákonodárného orgánu (poslanecká + říšská rada)
& Ukotvení práv občanů, zmínka o \textbf{duševní, hospodářské a národnostní svobodě}.
& Říšský (Ústavní) soud.
& Nezávislá soudní moc.
& Výkonná moc, ta je podřízena říšské radě.
& Dualismus Rakouska a Uherska (společná finanční a vojenská politika).
\end{easylist}

Postupně jsou u nás rozšiřována práva. R. 1907 bylo zevedeno \textbf{všeobecné rovné hlasovací právo pro muže}.

\subsection{2. období (1918--1992)}
 \paragraph{Prozatimní ústava (1918)} Tato ústava vzniká jako důsledek rozpadu Rak-Uhr., je pouze dočasná, intenzivně se pracuje na lepší ústavě.
 
 Tato ústava upravuje činnost \emph{Revolučního národního zhromáždění} (zákonodárného orgánu), prezidenta a vlády.
 
 \paragraph{Únorová ústava (1920)} Tato ústava byla značně inspirována Francouzskou ústavou, avšak ČSR přistupuje k systému parlamentní republiky.  Obsahovala \textbf{Preambuli} a \textbf{6 hlav}
 \begin{itemize}
 \item Zákonodárná moc je reprezentována dvoukomorovým Národním shromážděním (PS + Senát\footnote{Funkce senátu se tehdy příliš neosvědčila.}). 
 \item Výkoknná moc je reprezentována vládou a prezidentem (T. G. Masaryk), funkční období bylo 7 leté\footnote{Podobné pravomoce jako dnes, avšak prezident alespoň jednou do roka musí předsoupit před zákonodárný orgán a podávat správu o stavu republiky, měl právo \textbf{kontrasignace} a \textbf{suspenzivního veta}.}.
 \item Vláda je kolektivní orgán výkonné moci, odpovídá PS, která jí může vyslovit nedůvěru.
 \item Soudní moc je rozdělena do více soudů (Volební ssoud, Státní soud pro těžké zločiny, Vojenské soudy\dots)
 \item Všeobecné volební právo pro ženy.
 \end{itemize}

\paragraph{Mnichovská smlouva (září 1938)} Končí suverenita ČSR. Slovensko vyhlašuje autonomii (klerofašistický štát). Vyhlášení protektorátu Čechy a Morava. Existují exilové vlády. V srpnu 1944 E. Beneš vyhlašuje obnovení ústavního pořádku. 

\paragraph{Poválečný vývoj} Po 2. světové válce získává vliv sovětský svaz, r. 1946 se konají volby do NS, ty vyhrává KSČ, která se postupně dostává k moci (25. února). E. Beneš rezignuje a později umírá.

\paragraph{Ústava 9. května (1948)} První ústava umožňující porušování lid. práv (právo na soukromý majetek). Zavádí národní výbory (radnice\dots), vzniká \textbf{Slovenská národní rada}, zavedeno plánované hospodářství.

\paragraph{Socialistická ústava (1960)}Byla schválena protže bylo konstatováno vybudování socializmu na úzeí ČR. Odtud pochází název státu Československá socialistická republika.

Podle \textbf{článku č. 4} má v ČSR vedoucí úlohu Komunistická strana.

Její součástí je \textbf{zákon o Československé federaci}, ten říká že ČSSR je státem složený ze dvou rovnoprávných států, České soc. rep. a Slovenské soc. rep. Každý stát měl vlastní vládu a parlament. V rámci federace je zákonodárným orgánem Federální shromáždění\footnote{FS se skládalo ze dvou komor Sněmovny lidu a Sněmovny národů.}.

\paragraph{Polistopadový vývoj} Byl odstraněn čl. č. 4. Budue se právní stát a tržní hospodářství. Oficiálním názvem se ustálil \textbf{Česká a Slovenská federativní republika}. 1. září 1992 vyhlašuje Slovenská národní rada Ústavu Slovenské republiky. V listopadu 1992 byl schválen ústavní zákon o zániku ČSFR. 16. prosince 1992 schvaluje Česká národní rada Ústavu ČR.\footnote{Dohodlo se nástupnictví ČSFR, rozpad proběhl v klidu a míru.}

\subsection{3. období (1993--  )}

%\usepackage[top=1.5cm, bottom=1.5cm, outer=5cm, inner=2cm, heightrounded, marginparwidth=2.5cm, marginparsep=2cm]{geometry}
\subsection{Struktura ústavy}
Ústava se skládá z preambule a 8 hlav.
\begin{easylist}[enumerate]
\ListProperties(Style2*=,Numbers=a,Numbers1=R,FinalMark={.})
& Základní ustanovení
& Moc zákonodárná
& Moc výkonná
& Moc soudní
& Nejvyšší kontrolní úřad
& Česká národní banka
& Územní samospráva
& Závěrečná a přechodná ustanovení
\end{easylist}
Součástí ústavního pořádku ČR je také \textbf{Listina základních práv a svobod}.

\section{Preambule}
\begin{center}\textsl{
My, občané České republiky v Čechách, na Moravě a ve Slezsku,\\
v čase obnovy samostatného českého státu,\\
věrni všem dobrým tradicím dávné státnosti zemí Koruny české i státnosti československé,\\
odhodláni budovat, chránit a rozvíjet Českou republiku\\
v duchu nedotknutelných hodnot lidské důstojnosti a svobody\\
jako vlast rovnoprávných, svobodných občanů,\\
kteří jsou si vědomi svých povinností vůči druhým a zodpovědnosti vůči celku,\\
jako svobodný a demokratický stát, založený na úctě k lidským právům a na zásadách občanské společnosti,\\
jako součást rodiny evropských a světových demokracií,\\
odhodláni společně střežit a rozvíjet zděděné přírodní a kulturní, hmotné a duchovní bohatství,\\
odhodláni řídit se všemi osvědčenými principy právního státu,\\
prostřednictvím svých svobodně zvolených zástupců přijímáme tuto Ústavu České republiky.}
\end{center}

\paragraph{Zapamatovat si:} Odkaz na to že ústava vychází z tradic a kultury nejen českých nýbrž i evropských.

\section[I. hlava]{I. Základní ustanovení}
\begin{center} článek 1
\end{center}
\begin{enumerate} 
\item Česká republika je svrchovaný, jednotný a demokratický právní stát založený na úctě k právům a svobodám člověka a občana.
\item Česká republika dodržuje závazky, které pro ni vyplývají z mezinárodního práva.

\end{enumerate} \begin{center}
Článek 2
\end{center}
\begin{enumerate} \item Lid je zdrojem veškeré státní moci; vykonává ji prostřednictvím orgánů moci zákonodárné, výkonné a soudní.
\item Ústavní zákon může stanovit, kdy lid vykonává státní moc přímo.
\item Státní moc slouží všem občanům a lze ji uplatňovat jen v případech, v mezích a způsoby, které stanoví zákon.
(4) Každý občan může činit, co není zákonem zakázáno, a nikdo nesmí být nucen činit, co zákon neukládá.
\end{enumerate} \begin{center} článek 3\end{center}
Součástí ústavního pořádku České republiky je Listina základních práv a svobod.
 \begin{center} článek 4\end{center}
Základní práva a svobody jsou pod ochranou soudní moci.
\begin{center}  článek 5 \end{center}
Politický systém je založen na svobodném a dobrovolném vzniku a volné soutěži politických stran respektujících základní demokratické principy a odmítajících násilí
jako prostředek k prosazování svých zájmů.
 \begin{center} článek 6\end{center}
Politická rozhodnutí vycházejí z vůle většiny vyjádřené svobodným hlasováním. Rozhodování většiny dbá ochrany menšin.
 \begin{center}  Článek 7
Stát dbá o šetrné využívání přírodních zdrojů a ochranu přírodního bohatství.


\end{center}
 \begin{center}
Článek 8
\end{center}
Zaručuje se samospráva územních samosprávných celků.
 \begin{center}
Článek 9
\end{center}
\begin{enumerate} \item Ústava může být doplňována či měněna pouze ústavními zákony.
\item Změna podstatných náležitostí demokratického právního státu je nepřípustná.
\item Výkladem právních norem nelze oprávnit odstranění nebo ohrožení základů demokratického státu.
\end{enumerate} \begin{center}
Článek 10
\end{center}
Vyhlášené mezinárodní smlouvy, k jejichž ratifikaci dal Parlament souhlas a jimiž je Česká republika vázána, jsou součástí právního řádu; stanoví-li mezinárodní
smlouva něco jiného než zákon, použije se mezinárodní smlouva.
 \begin{center}
Článek 10a
\end{center}
\begin{enumerate} \item Mezinárodní smlouvou mohou být některé pravomoci orgánů České republiky přeneseny na mezinárodní organizaci nebo instituci.
\item K ratifikaci mezinárodní smlouvy uvedené v odstavci 1 je třeba souhlasu Parlamentu, nestanoví-li ústavní zákon, že k ratifikaci je třeba souhlasu daného v
referendu.
\end{enumerate} \begin{center}
Článek 10b
\end{center}
\begin{enumerate} \item Vláda pravidelně a předem informuje Parlament o otázkách souvisejících se závazky vyplývajícími z členství České republiky v mezinárodní organizaci nebo
instituci uvedené v čl. 10a.
\item Komory Parlamentu se vyjadřují k připravovaným rozhodnutím takové mezinárodní organizace nebo instituce způsobem, který stanoví jejich jednací řády.
\item Zákon o zásadách jednání a styku obou komor mezi sebou, jakož i navenek, může svěřit výkon působnosti komor podle odstavce 2 společnému orgánu komor.
\end{enumerate} \begin{center}
Článek 11
\end{center}
Území České republiky tvoří nedílný celek, jehož státní hranice mohou být měněny jen ústavním zákonem.
 \begin{center}
Článek 12
\end{center}
\begin{enumerate} \item Nabývání a pozbývání státního občanství České republiky stanoví zákon.
\item Nikdo nemůže být proti své vůli zbaven státního občanství.
\end{enumerate} \begin{center}
Článek 13
\end{center}
Hlavním městem České republiky je Praha.
 \begin{center}
Článek 14
\end{center}

\begin{enumerate} \item Státními symboly České republiky jsou velký a malý státní znak, státní barvy, státní vlajka, vlajka prezidenta republiky, státní pečeť a státní hymna.
\item Státní symboly a jejich používání upraví zákon.
\end{enumerate}
%%%%%%%%%%%%%%%%%%%%%%%%%%%%%%%%

\section[II. hlava]{II. Zákonodárná moc}
Týká se navrhování a schvalování zákonů a je představována \textbf{dvoukomorovým parlamentem}.
\paragraph{Poslanecká sněmovna:} 200 poslanců, pasivní volební právo od 21 let neodvolatelnost, poměrný volební systém, zasedání jsou veřejná, poslanci mají možnost interpelací (t.j. vznést odborný dotaz na člena vlády, ti jsou povinni do 30 dnů odpovědět)

\paragraph{Senát:} 81 členů, od 40 let většinová volba


Mandát je získán volbou, úřadu se člověk ujímá složením slibu. Pokud je slib složen s výhradami neo není složen pak mandát neplatí.

\begin{center}
\textsf{Funkce senátora a poslance jsou neslučitelné !!}
\end{center}

\paragraph{Imunita:} Vzniká proto aby se členové nebáli následků za své konaní. Poslanci nemohou být stíháni za výroky pronesené na půdě Parlamentu. Aby mohl být trestně stíhán musí ho komora vydat ke stíhání.

\paragraph{Indemnita:}poslanci ani senátoři nejsou stíháni za výroky pronesené mimo hlavní jednací sál, ale v budově parlamentu

\ust{53}{\begin{enumerate}
\item Každý poslanec má právo interpelovat vládu nebo její členy ve věcech jejich působnosti.
\item Interpelovaní členové vlády odpovědí na interpelaci do třiceti dnů ode dne jejího podání.\footnote{ Otázka musí být opravdu kvalifikovaná a týkat se daného resortu.}
\end{enumerate}}

\paragraph{Rozpuštění sněmovny} Pouze \textbf{Prezident republiky} může rozpustit sněmovnu, a to z důvodů:
\begin{itemize}
\item Pokud není ani na návrh předsedy PS vyslovena důvěra vládě
\item Pokud s rozpuštěním souhlasí 3/5 \textbf{všech} (!!!!!!) poslanců.
\item Pokud není přijat zákon spjatý s důvěrou vlády. 
\end{itemize}
Po rozpuštění pravomoce přebírá \textbf{Senát}, avšak nesmí dělat změny v \textbf{ústavě, příjmat mezinárodní  smlouvy, rozhodovat o státním rozpočtu.} Zákony zpětně musí být schváleny PS.

\begin{footnotesize}
\ust{35}{\begin{enumerate}

\item Poslaneckou sněmovnu může rozpustit prezident republiky, jestliže \begin{enumerate}

\item Poslanecká sněmovna \textbf{nevyslovila důvěru nově jmenované vládě}\footnote{Nedošlo k vyslovení důvěry na 3. pokus.}, jejíž předseda byl prezidentem republiky jmenován na návrh předsedy Poslanecké sněmovny,
\item Poslanecká sněmovna se neusnese do tří měsíců o vládním návrhu zákona, s jehož projednáním spojila vláda otázku důvěry,
\item zasedání Poslanecké sněmovny bylo přerušeno po dobu delší, než je přípustné,
\item Poslanecká sněmovna nebyla po dobu delší tří měsíců způsobilá se usnášet, ačkoliv nebylo její zasedání přerušeno a ačkoliv byla v té době opakovaně svolána ke schůzi. 
\end{enumerate}
\item Prezident republiky Poslaneckou sněmovnu rozpustí, navrhne-li mu to Poslanecká sněmovna usnesením, s nímž vyslovila souhlas \textbf{třípětinová většina všech poslanců}. 
\item Poslaneckou sněmovnu nelze rozpustit tři měsíce před skončením jejího volebního období.
\end{enumerate}}

\ust{33}\begin{enumerate}

\item Dojde-li k rozpuštění Poslanecké sněmovny, přísluší Senátu přijímat zákonná opatření ve věcech, které nesnesou odkladu a vyžadovaly by jinak přijetí zákona.
\item Senátu však nepřísluší příjímat zákonné opatření ve věcech Ústavy, státního rozpočtu, státního závěrečného účtu, volebního zákona a mezinárodních smluv podle čl. 10.
\item Zákonné opatření může Senátu navrhnout jen vláda.
\item Zákonné opatření Senátu podepisuje předseda Senátu, prezident republiky a předseda vlády; vyhlašuje se stejně jako zákony.
\item Zákonné opatření Senátu musí být schváleno Poslaneckou sněmovnou na její první schůzi. Neschválí-li je Poslanecká sněmovna, pozbývá další platnosti.
\end{enumerate}
\end{footnotesize}

\begin{description}
\item[Nadpoloviční kvórum] to je zapotřebí při rozhodování o věcech armády (zahraniční mise, válečný stav, pohyb cizích vojsk na území ČR). Je zapotřebí nadpoloviční většina všech Poslanců a Senátorů.
\item[Kvalifikované kvórum] je zapotřebí pro změny ústavy viz b. 4 čl. 39. 3/5 všech poslanců a 3/5 přítomných senátorů.
\end{description}

\ust{39}\begin{footnotesize}
\begin{enumerate}
\item1 Komory jsou způsobilé se usnášet za přítomnosti alespoň \textbf{jedné třetiny} svých \textbf{členů}.
\item K přijetí usnesení komory je třeba souhlasu\textbf{ nadpoloviční většiny přítomných} poslanců nebo senátorů, nestanoví-li Ústava jinak.
\item K přijetí usnesení o vyhlášení \textit{válečného stavu} a k přijetí usnesení o souhlasu s vysláním \textit{ozbrojených sil} České republiky mimo území České republiky nebo s pobytem ozbrojených sil jiných států na území České republiky, jakož i k přijetí usnesení o \textit{účasti České republiky} v obranných systémech \textit{mezinárodní organizace}\footnote{Např v NATO nebo OSN.}, jíž je Česká republika členem, je třeba souhlasu \textbf{nadpoloviční většiny všech} poslanců a nadpoloviční většiny všech senátorů. 
\item K přijetí \textbf{ústavního zákona} a souhlasu k ratifikaci \textbf{mezinárodní smlouvy} uvedené v čl. 10a odst. 1 je třeba souhlasu \textbf{třípětinové většiny všech} poslanců a \textbf{třípětinové většiny přítomných} senátorů.
\end{enumerate}
\end{footnotesize}

\subsection{Legislativní proces}
\begin{enumerate}
\item Návrhy zákonů se podávají Poslanecké sněmovně (poslanec, skupina poslanců, Senát, vláda nebo zastupitelstvo vyššího územního samosprávného celku)
\begin{footnotesize}
\ust{42}{
(1) Návrh zákona o státním rozpočtu a návrh státního závěrečného účtu podává vláda.\\
(2) Tyto návrhy projednává na veřejné schůzi a usnáší se o nich jen Poslanecká sněmovna}
\end{footnotesize}
\item Vyjádření vlády (má 30 dnů na vyjádření)

%Vláda má do 30 dnů právo vyjádřit se k zákonům, jinak je bráno kladné vyjádření. 

%Vláda je oprávněna žádat, aby Poslanecká sněmovna skončila projednávání vládního návrhu zákona do tří měsíců od jeho předložení, pokud s tím vláda spojí žádost o vyslovení důvěry.
\item PS usnáší o zákonu. (probíhají 3 čtení, během nich je možnost na připomínky a pozměňovací návrhy).
\begin{itemize}
\item jestliže zákon není schválen tak \textsf{proces končí}
\end{itemize}

\item Po odsouhlasení zákona je předán Senátu.

\item Senát má 30 dnů na přijetí zákona. Pokud se nevyjádří je situace chápána že zákon je schválen.
\begin{itemize}
\item Pokud Senát celý zákon zamítne a vrátí ho do PS pokud poslanci chtějí i přesto podpořit zákon je zapotřebí podpora 101 poslanců. Pokud se tak nestane \textsf{proces končí.}
\end{itemize}
\item Po schválení nebo přehlasování podstupuje zákon Prezidentovi, ten může využít práva \emph{suspenzivního veta}.
\begin{itemize}
\item Po využití práva veta se zákon vrací do PS, pak je postup stejný jako by se vracel ze senátu (\textbf{101 poslanců !!}).
\end{itemize}
\item Zákon je podepsán Prezidentem, Premiérem a předsedou PS

\item Vyhlášení ve sbírce zákonů s uvedením vstoupení v platnost.
\end{enumerate}


\section[III. hlava]{III. Výkonná moc}
Moc vykonávající zákony. Je představována Prezidentem a Vládou.
\subsection{Prezident}
Na Prezidenta ČR může kandidovat každý občan ČR, který je volitelný do Senátu maximálně dvě volební období po sobě. Není z výkonu své funkce odpovědný.

Na prezidenta republiky může být navržen kandidát navržen: \begin{itemize}
\item Občanem ČR s peticí minimálně 50\,000 podpisů.
\item Nejméně 10 Senátorů.
\item Nejméně 20 Senátorů.
\end{itemize}

Volba je dvoukolová, pokud v 1. kole nikdo nezíská 50\%, tak se do 14 dnů po prvním kole konná 2. kolo voleb, pokud dojde k remíze celá volba se opakuje. Prezident skládá slib do rukou předsedy Senát. Prezident může být trestně stíhán pouze za \textbf{vlastizradu}, pak potřebuje být vydán Senátem a pak je souzen Ústavním soudem.

\paragraph{Pravomoce prezidenta:}
\begin{description}
\item[Zákonodárná moc] -- zasahuje do zákonodárné moci

\begin{itemize}
\item svolává a rozpouští PS,
\item může se zúčastnit jednání obou komor Parlamentu
\item podepisuje zákony
\item vyhlašuje volby do komor parlamentu
\item může vrátit zákon zpátky do PS kromě ústavních zákonů
\end{itemize}
\item[Výkonná moc] -- prezident vykonává

\begin{itemize}
\item jmenuje a přijímá demisi
\item jmenování ministrů, premiéra
\item demise premiéra-za celou vládu prezidentovi
\item demise ministra-prostřednictvím premiéra prezidentovi $\rightarrow$ nahradí se jiným
\end{itemize}
\item[Soudní moc:]o soudech rozhoduje
  
\begin{itemize}
\item za stavení trestního stíhání
\item jmenuje soudce
\item právo udělit milost, normálně pro jednoho, amnestii při významných událostech, pro více lidí může zmírnit trest
\end{itemize}
\item[jmenuje:] Profesory, Generály, Vysoké úředníky
\item[Dále:]Reprezentace navenek, Velitel ozbrojených sil
\end{description}
\subsection{Vláda}
\begin{center}
Vláda je vrcholným orgánem výkonné moci.
\end{center}

Vláda je odpovědna PS, musí získat její důvěru (50 poslanců může vyvolat hlasování o důvěře).

\paragraph{Jmenování:} Premiér je pověřen prezidentem pověřením vlády. Podle toho kolik stran je součástí vlády jsou vlády:
\begin{description}
\item[Jednobarevná:]vláda se skládá z 1 strany
\item[Vícebarevná (koaliční)] skládá se z více stran.
\item[Menšinová vláda:] vláda nemá 101 hlasů ve sněmovně pak potřebuje podporu
\item[Většinová vláda:] má více než 101 hlasů ve sněmovně.
 
\end{description}

Nově jmenovaná vláda má 30 dnů na vytvoření programového prohlášení a žádá PS o důvěru, pokud ji nezíská prezident má ještě jeden pokus na jmenování premiéra, třetího premiéra jmenuje Prezident na návrh předsedy PS.

\section[IV. hlava]{IV. Soudní moc}
Je přestavována systémem nezávislých soudů, nezávislých na zákonodárné a výkonné moci. Dohlíží na dodržování zákonů.

\paragraph{Soudce:} nesmí být členem žádné politické strany ani představitelem jiné státní moci.

\subsection{Ústavní soud}
Ústavní soud je soudním orgánem ochrany ústavnosti.

Soudcem ÚS může být občan který,\begin{itemize}
\item má právnické vzdělání a 10 let praxe v oboru,
\item volitelný do Senátu.
\end{itemize}

Do úřadu je jmenován Prezidentem se souhlasem Senátu na dobu 10 let. Ústavních soudců může být 15. V čele stojí předseda, dva místopředsedové a 4 tříčlenné senáty.\footnote{Některými problémy se zabývá všech 15 soudců.}

\begin{wrapfigure}{r}{5cm}
\includegraphics[width=5cm]{rychetsky}
\caption{Pavel Rychetský, předseda ústavního soudu.}
\end{wrapfigure}

Znát 3 pravomoce ÚS. 
\begin{itemize}
\item  Ochrana ústavnosti -- v případě napadnutí zákona rozhoduje rozporu právních předpisů s ústavou
\end{itemize}

\begin{small}
\ust{87}
\begin{enumerate}
\item Ústavní soud rozhoduje\begin{enumerate}

\item o zrušení zákonů nebo jejich jednotlivých ustanovení, jsou-li v rozporu s ústavním pořádkem,
\item o zrušení jiných právních předpisů nebo jejich jednotlivých ustanovení, jsou-li v rozporu s ústavním pořádkem nebo zákonem, 
\item o ústavní stížnosti orgánů územní samosprávy proti nezákonnému zásahu státu,
\item o ústavní stížnosti proti pravomocnému rozhodnutí a jinému zásahu orgánů veřejné moci do ústavně zaručených základních práv a svobod,
\item o opravném prostředku proti rozhodnutí ve věci ověření volby poslance nebo senátora,
\item v pochybnostech o ztrátě volitelnosti a o neslučitelnosti výkonu funkcí poslance nebo senátora podle čl. 25,
\item o ústavní žalobě Senátu proti prezidentu republiku podle čl. 65 odst. 2,
\item o návrhu prezidenta republiky na zrušení usnesení Poslanecké sněmovny a Senátu podle čl. 66,
\item o opatřeních nezbytných k provedení rozhodnutí mezinárodního soudu, které je pro Českou republiku závazné, pokud je nelze provést jinak,
\item o tom, zda rozhodnutí o rozpuštění politické strany nebo jiné rozhodnutí týkající se činnosti politické strany je ve shodě s ústavními nebo jinými zákony,
\item spory o rozsah kompetencí státních orgánů a orgánů územní samosprávy, nepřísluší-li podle zákona jinému orgánu. 
\end{enumerate}
\item Ústavní soud dále rozhoduje o souladu mezinárodní smlouvy podle čl. 10a a čl. 49 s ústavním pořádkem, a to před její ratifikací. Do rozhodnutí Ústavního soudu nemůže být smlouva ratifikována.
\item Zákon může stanovit, že namísto Ústavního soudu rozhoduje Nejvyšší správní soud \begin{enumerate}

\item o zrušení právních předpisů nebo jejich jednotlivých ustanovení, jsou- li v rozporu se zákonem,
\item spory o rozsah kompetencí státních orgánů a orgánů územní samosprávy, nepřísluší-li podle zákona jinému orgánu.

\end{enumerate}
\end{enumerate}
\end{small}

\subsection{Soustava soudů}

\subsection{Dokumenty o Lidských právech}

\paragraph{Magna charta liber tatum(1215)} první dokument o lidských právech, omezuje pravomoc panovníka vůči lidem.

\paragraph{Deklarace práv člověka a občana:} poprvé se mluví o \textbf{rovnosti lidí v právech a svobodě}.Dalšími principy jsou: \begin{itemize}
\item presumpce neviny
\item svoboda vyznání
\item \dots
\end{itemize}

\paragraph{Charta OSN (1945):}Tato chart zdůrazňuje úctu lid práv bez rozdílu pohlaví vyznání \dots

\paragraph{Všeobecná deklarace lid. práv:} V tomto dokumentu se zdůrazňují sociální práva (odměna za práci, odpočinek \dots)

\paragraph{Konference o bezpečnosti a spolupráci v Evropě(Helsinky 1975):} zde se sešli zástupci všech evrop. zemí s zástupci USA a Kanady, všichni zástupci podepsali \textsc{Závěrečný akt} a zavázali se k dodržování lid. práv

\paragraph{Charta 77} dokument na protest proti potlačování lid. práv, autoři: V. Havel, Jan Patočka, Jiří Hájek

\paragraph{Amnesty International} mezinárodní nezávislá organizace zabývající se lid. právy. Byla založena roku 1961 Brit. právníkem kvůli odsouzení portugalských studentu za přípitek na svobodu.

\section[V. hlava]{V. Nejvyšší kontrolní úřad}
Nezávislý kontrolní orgán dohlížející na hospodaření se státním majetkem a plnění stát rozpočtu. 

\section[VI. hlava]{VI. Česká národní banka}
ČNB je ústřední banka státu, dohlíží na stabilitu měny, jejími doporučeními by se měla řídit vláda. V čele  je Bankovní rada, jmenována Prezidentem. V čele stojí Guvernér.

\section{VII. Územní samospráva}
Územní samospráva je dvoustupňová. Vyšší územně samosprávné celky (kraje), nižší územně samosprávné celky (obce).

\section{VIII. Přechodná a závěrečná ustanovení}

\section{Listina základních práv a svobod}
Má 6 hlav, je součástí ústavního pořádku ČR.
\subsection{Obecná ustanovení}
 Základní práva a svobody jsou nezadatelné\footnote{Znamená, že práva nejsou výtvorem státní moci, ale stát je musí uznat chce-li být vnímán jako právní.}, nezcizitelné\footnote{Nelze je nikomu odebrat.}, nepromlčitelné\footnote{Jsou trvale vymahatelná.} a nezrušitelné.
 
 \subsection{Lidská práva a základní svobody}
Základním právem je právo na život. Lidé na svých právech mohou být omezeni pouze zákonem.  

\subsubsection{Základní lidská práva a svobody}

Obviněného nebo podezřelého z trestného činu je možno zadržet jen v případech stanovených v zákoně. Zadržená osoba musí být ihned seznámena s důvody zadržení, vyslechnuta a nejpozději do 48 hodin propuštěna na svobodu nebo odevzdána soudu. Soudce musí zadrženou osobu do 24 hodin od převzetí vyslechnout a rozhodnout o vazbě, nebo ji propustit na svobodu.

Každý má právo vlastnit majetek, avšak pokud je to ve veřejném zájmu má stát právo danou osobu vyvlastnit. 

Je dáno právo na nedotknutelnost obydlí a listovního tajemství.

 Svoboda myšlení, svědomí a náboženského vyznání je zaručena. Každý má právo změnit své náboženství nebo víru anebo být bez náboženského vyznání.
 s
\begin{center}
\textsf{Znát 4 práva z této hlavy!!}
\end{center}
\subsubsection{Politická práva}

Právo se shromažďovat a svobodně projevovat a svobodně získávat informace.Právo zakládat politické strany.

\subsubsection{Soudní práva}
\begin{itemize}
\item Každý má právo na spravedlivý soudní proces.
\item Svých práv se můžu domáhat pokud jsem na nich poškozen rozhodnutím státní správy.
\item Každý má právo na právní pomoc v řízení před soudy.
\item\textsf{\textbf{Je ctěna presumpce neviny.}} 
\end{itemize}
Typy právnických profesí:
\begin{description}
\item[Soudce] soudí.
\item[Státní zástupce]hájí zájmy státu, vznáší obžalobu a navrhuje výši trestu.
\item[Notář] do své funkce je jmenován ministrem spravedlnosti. Může poskytovat právní radu, ověřovat pravost některých dokumentů. Dohlíží na průběh soutěží. Vystupuje jako \textsf{soudní komisař} v otázkách dědictví.\footnote{Jejich počet je stanoven na spát pod působnost určitého soudu.}
\item[Advokát] je osoba, která musí mít slouženou advokátní zkoušku. Ten zastupuje a hájí své klienty. Pokud si ho někdo nemůže dovolit je člověku \emph{ex offo} přidělen.
\end{description}
\chapter{Úvod do práva}
Právo vzniká už v antice, protože lidé nedodržovali závazky, tak aby fungovala společnost.
\begin{quote}\textsf{
Právo je souhrn právních norem platný na území daného státu a státní mocí vynutitelný}
\end{quote}
•	Ve společnosti existují právní normy, morální normy, náboženské

o	Morální normy nejsou psané, předávají se výchovou, není za ně žádný trest jen výčitky svědomí

o	Právní psané, regulují chování lidí, aby si nedělali, co chtějí, za jejich porušení trest (např. odnětí svobody), závazné pro všechny (cizinci také, na krátké návštěvě, dlouhodobém pobytu)

•	Každý stát si vytváří specifické právní normy (co je někde povoleno, může být jinde zakázáno-potraty, trest smrti, polygamie)

\subsubsection*{Funkce práva}
\begin{description}
\item[Kontrolní]
\item[Právní jistota] člověk ví co může od svého jednání očekávat.
\end{description}

\section{Právní vztahy}
Jsou to spol. vztahy upravené právními normami. 
Tyto vztahy tvoří:
\begin{description}
\item[Účastnící] právních vztahů, jsou fyzické či právnické osoby
\begin{itemize}
\item Fyzická osoba vzniká narozením\footnote{V určitých situacích může být jako fyz osoba chápáno již počaté dítě.}, končí úmrtím či úředním prohlášením za mrtvého. Každá fyz osoba má \textbf{právní osobnost}, tzn. \textsf{způsobilost mít práva a povinnosti} a \textbf{svéprávnost}, tj. \textsf{způsobilost k právním úkonům}. Každá fyz. osoba má právo na ochranu osobnosti, osobních údajů, zdraví, života a majetku.

\item Právnická osoba, je uměle vytvořený subjekt (fyz. osoba, více fyz. osob, instituce, společnost,\dots). Stát je také práv. osobou. Vznikají smlouvou nebo zakládací listinou, musí být zaevidovány (obch. rejstřík) zde musí být název a sídlo společnosti.
\end{itemize}

\item[Obsah] jsou práva a povinnosti účastníků.

\item[Předmět] je cíl ke kterému právní vztahy směřují.

\end{description}
\section{Rezdělení práva}
\subsection{Podle místa působnosti}
\begin{description}
\item[Mezinárodní právo] upravuje vztahy mezi jednotlivými zeměmi.
\item[Vnitrostátní právo] si vypracoval každý stát pro své právní potřeby.
\end{description}
\subsection{Veřejné a soukromé právo}
\begin{description}
\item[Veřejné právo] upravuje vztahy mezi fyz. a práv osobami a státem. Např. \textsf{ústavní právo, správní právo, trestní právo, finanční právo} (daně,\dots)
\item[Soukromé právo] upravuje vztahy mezi fyz. a právnickými osobami.
\textsf{občanské právo},(upravuje především majetkové vztahy mezi osobami), \textsf{rodinné právo, obchodní právo}
\end{description}
\textsf{Pracovní právo} tvoří vyjímku a patří do oubou skupin.

\paragraph{Poznámky:}

\dotfill{} \vspace{0.2cm}

\dotfill{} \vspace{0.2cm}

\dotfill{} \vspace{0.2cm}

\dotfill{} \vspace{0.2cm}

\dotfill{} \vspace{0.2cm}

\dotfill{} \vspace{0.2cm}

\dotfill{} \vspace{0.2cm}

\dotfill{} \vspace{0.2cm}

\dotfill{} \vspace{0.2cm}

\dotfill{} \vspace{0.2cm}

\dotfill{} \vspace{0.2cm}

\dotfill{} \vspace{0.2cm}

\dotfill{} \vspace{0.2cm}

\dotfill{} \vspace{0.2cm}

\dotfill{} \vspace{0.2cm}

\dotfill{} \vspace{0.2cm}

\dotfill{} \vspace{0.2cm}

\dotfill{} \vspace{0.2cm}

\dotfill{} \vspace{0.2cm}

\dotfill{} \vspace{0.2cm}


\chapter{Občanské právo}
Občanské právo hlavně upravuje přvážně majetkové vztahy mezi jednotlivými osobami. Občanské právo vychází z ústavního práva.

Důležitou součástí je \textbf{Občanský zákoník} ten byl vydán roku 2012 a vstoupil v platnost r. 2014. Skládá se ze 4 částí:\begin{enumerate}
\item Popisuje kdo je fyzická a právnická osoba (zmínky o jménu a příjmení,smrt,\dots)
\item Rodinné právo
\item Majetkové právo (nabytí, pozbytí majetku,\dots)
\item Smlouvy, závazkové právo
\item Závěrečná ustanovení
\end{enumerate}

\section{Majetkové právo}
Vlastnictví znamená, že vlastník má právo věc držet, nakládat s ní, používat její plody, 

Vlastnictví lze získat, ztratit dvma způsoby: \begin{description}
\item[Převedem]kopě, prodej, darování
\item[Přechodem] díky dědictví. 
\end{description}

Součástí vlastnictví mohou být věci jak hmotné tak nehmotné.

\section{Rodinné právo}

\subsection{Vyživovací povinnost mezi rodiči a dětmi}

Matka a otec dítěte jsou jejich biologickými rodiči. Po narození jsou rodiče povinni postarat se jak materiálně tak citově o dítě. Rodiče jsou povinni se starat o své děti dokud studují nejpozději však do 26 let.

\subsection{Náhradní rodinná péče}
Vždy o náhradní rodinné péči rozhoduje soud.

Babybox -- dítě nesmí mít u sebe doklady (jinak hrozí rodičům trestní stíhání).

\paragraph{Individuální rodinná péče}\begin{itemize}

\item \textbf{Osvojení}\footnote{V ČR je o něj poměrně velký zájem.} osvojitelé musí být vždy manželé, u párů je monitorováno prostředí, dítě je jim svěřeno a následně dochází k rozhodnutí o svěření do péče (rok), dítě je právně brán jako člen rodiny (dědictví...), lze osvojit i dospělou osobu
\item \textbf{Pěstounská péče}, pěstoun může být i jeden, pěstouni dostávají odměnu od státu, častokrát se do ní dostávají děti nevhodné pro osvojení, děti jsou v péči pouze do 18 let
\item \textbf{Poručnictví} spec. forma rod. péče, vzniká kvůli neschopnosti rodičů se starat o děti. Pokud někdo z jejich blízkého okolí projeví zájem postarat se i dítě, pak se o něj může starat. 
\end{itemize}

\paragraph{Kolektivní}\begin{itemize}
\item Kojenecký ústav do 2 let 
\item Dětský domov od 2 let pokud se nenajde pěstoun
\item SOS dětské vesničky, klokánek -- dočasné zařízení
\end{itemize}

\section{Závazkové právo}

Týká se závazků, nejčastěji smluv.

\subsection{Smlouvy}

Musí být uzavřeny dobrovolně, obě strany mají právo rozumnět obsahu smlouvy. Uzavření smlouvy má právní váhu. Může být uzavřena písemně nebo ústně, pouze některé smlouvy (stanoveno zákonem) musí mít písemnou podobu.

\paragraph{Konkludentní} smlouva je taková smlouva ze které je patrný účel uzavření smlouvy.

Občanský zákoník dnes straní slabším stranám.

Smlouva která byla uzavřena

\paragraph{Kupní smlouva} je smlouva, u které prodávající převádí vlastnická práva na kupujícího. ze smlouvy musí být jasné co bylo prodáno.

\paragraph{Darovací smlouva} darující převádí vlastnická práva na obdarovaného, nesmí obsahovat podmínku společenské úsluhy. Pokud obdarovaný nezachází s věcí podle záměrů dárce tak ji může chtít zpět.

\paragraph{Nájemní smlouva} Pronajímatel za nájemné pronajímá majetek nájemci. Nájemce se o věc musí pravidelně starat. Pronajímatel si může určit podmínky pronájmu.


\section{Dědictví}
Převedení majetkových práv ze zůstavitele (nebožtíka) na pozůstalého. Dědit může fyzická i právnická osoba\footnote{Dědit může i práv osoba, která ještě nevznikla ale do roka vznikne.}. Součástí dědictví jsou všechny dluhy (na začátku se může dědic vzdát práva dědit).

Lze dědit:
\begin{description}
\item[Dědická smlouva] při dědění smlouvou nelze odkázat veškerý majetek, ten člověk se pak nepočítá mezi dědice(nemusí hradit dluhy).
\item[Závěť]  K závěti by se mělo přistupovat jo k poslednímu prání zůstavitele. Závěť musí obsahovat datum (platí vždy nejmladší). \begin{itemize}
	\item \emph{Holografní závěť} -- sepsána vlastní rukou
	\item \emph{Alografní závěť} -- sepsaná jiným způsobem než vlastní rukou, je pořízena za přítomnosti dvou svědků (nejsou zainteresovaní na dědictví)
	\item Notářský zápis -- návštěva notáře následně se závěť nechá u notáře (1\,200 Kč)
	\end{itemize}
	
	\emph{Nepominutelný dědic} jsou všechny děti zůstavitele a mají právo na část dědictví.
	
\item[Veřejná listina] -- závěť sepsaná notářem

\item[Zákon] Pokud se neobjeví závěť, dědí se podle zákona.

\end{description} 

Pokud rodič nechce aby děti zdědily majetek může je \textbf{vydědit} a to pro následující důvody: \begin{itemize}
\item dítě bylo odsouzeno k trestu odnětí svobody v době trvání nejm. 1 roku na \emph{úmyslný} trest čin,
\item vede trvale nezřízený život (Alkoholizmus, prostituce\dots),
\item neposkytl pomoc, kterou jako potomek poskytnout měl,
\item potomek je velmi zadlužený a počíná si marnotrapně, že je obava o majetek
\end{itemize}

Pokud se nanajde žádný dědic, majetek připadá státu jako \emph{odúmrť}.

\chapter{Trestní právo}
Má za úkol ochránit majetek a také prevenci dalších nebezpečných jednání.

\section{Trestní právo hmotné}
Definuje trestné činny, kdo je pachatelem TČ, druhy TČ a druhy trestů.

\paragraph{Trestný čin} je charakteristický:
\begin{itemize}
\item míra společenské nebezpečnosti
\item je popsán v trestním řádě
\item je tam určitá míra zavinění.
\end{itemize}

\paragraph{Pachatel TČ} tím je FO nebo PO, za toho je považován člověk, který jej spáchal, pokusil se o něj, podílel se na něm (objednali si ho, naváděl k němu, či o něm věděl\dots)
\begin{description}
\item[Nezletilý] < 15 let, tato osoba nemůže být trestně stíhána, zákonný zástupce musí uhradit vzniklou škodu. Dítě je předáno do diagnostického ústavu.
\item[Mladistvý] 15 -- 18 let, mají základní trestní odpovědnost, dostávají poloviční sazbu trestu dospělých, můžou dostat maximálně 10 let, kvůli resocializaci 
\item[Dospělý] > 18 let, plná trestní odpovědnost
\end{description}

\paragraph{Druhy trestných činnů} soud může jednání kvalifikovat jako jednání v beztrestnosti,
\begin{itemize}
\item nesvéprávnost
\item věk
\item nutná obrana
\item jednání v krajní nouzi (například když jde o život)
\item oprávněné použití zbraně
\end{itemize}

\subsection{Tresty}
\begin{multicols}{2}
\begin{footnotesize}
\begin{enumerate}

\item Za spáchané trestné činy může soud uložit tresty

\begin{enumerate}
\item odnětí svobody,

\item domácí vězení,

\item obecně prospěšné práce,

\item propadnutí majetku,

\item peněžitý trest,

\item propadnutí věci,

\item zákaz činnosti,

\item zákaz pobytu,

\item zákaz vstupu na sportovní, kulturní a jiné společenské akce,

\item ztrátu čestných titulů nebo vyznamenání,

\item ztrátu vojenské hodnosti,

\item vyhoštění.

\end{enumerate}
\item Trestem odnětí svobody se rozumí, nestanoví-li trestní zákon jinak,
\begin{enumerate}

\item nepodmíněný trest odnětí svobody,

\item podmíněné odsouzení k trestu odnětí svobody,

\item podmíněné odsouzení k trestu odnětí svobody s dohledem.

\end{enumerate}
\item Zvláštním typem trestu odnětí svobody je výjimečný trest (§ 54).
\end{enumerate}
\end{footnotesize}
\end{multicols}

\paragraph{Odnětí svobody:} může být podmíněné či nepodmíněné
\begin{itemize}
\item pokud je člověk podmíněně odsouzen nejde do vězení a nesmí během podmímnky spáchat trestný čin

\item nepodmíněný trest, pahcatel natupuje výkon trestu, po výkonu části trestu lze požádat o podmíněné propuštění.
\end{itemize}
\paragraph{Polehčující okolnosti:} spolupráce, vyjádření lítosti, náhrada škody

\paragraph{Přitěžující okolnosti:}opakovaná trestná činnost, 


\section{Trestní právo procesní}
Upravuje postup orgánů činných v trestním řízení.

Orgány: Policie, soudy, státní zastupitelství.

\dotfill{} \vspace{0.6cm}

\dotfill{} \vspace{0.6cm}

\dotfill{} \vspace{0.6cm}

\dotfill{} \vspace{0.6cm}

\dotfill{} \vspace{0.6cm}

\dotfill{} \vspace{0.6cm}

\dotfill{} \vspace{0.6cm}

\dotfill{} \vspace{0.6cm}

\dotfill{} \vspace{0.6cm}


\dotfill{} \vspace{0.6cm}

\dotfill{} \vspace{0.6cm}

\dotfill{} \vspace{0.6cm}

\dotfill{} \vspace{0.6cm}

\dotfill{} \vspace{0.6cm}

\dotfill{} \vspace{0.6cm}

\chapter{Pracovní právo}
\begin{center}

Upravuje vztahy mezi zaměstnancem a zaměstnavatelem.
\end{center}

\begin{description}
\item[Zaměstnanec] je FO k práci způsobilá.

\item[Zaměstnavatel] je FO nebo PO
\end{description}

Zaměstnání lze získat\begin{description}
\item[volba]např člen Parlamentu ČR

\item[jmenování] je přímo někým jmenován

\item[smlouva] je nejčastější 
\end{description}

\paragraph{Pracovní smlouva} Musí obsahovat
\begin{itemize}
\item os. údaje zaměstnance
\item název a sídlo zaměstnavatele
\item \textbf{Místo práce}
\item \textbf{popis práce}
\item výše mzdy
\item druh pracovního poměru (automaticky se bere poměr na dobu neurčitou)
\item \textbf{datum}

\end{itemize}

\paragraph{Poměr na dobu neurčitou} je výhodnější pro zaměstnance


Před podpisem prac. smlouvy jsou obě strany seznámeny s právy a povinnostmi.

\paragraph{Práva a povinnosti zaměstnavatele} Musí vytvářet příznivé prac. prostředí, zadává práci, poskytovat zaměstnanci stravování...

Pracovní smlouva začíná daným datem běžet. Existuje zkušební lhůta během které lze bez udání důvodu ukončit prac. poměr.

Ukončení prac. poměru:
\begin{description}
\item[Událostí] smrt, krach společnosti

\item[Právním úkonem:]
\begin{description}
	\item[Jednostranný] výpověď, okamžité zrušení	
	\item[Dvoustranný] dohoda
\end{description}
\end{description}

\paragraph{Výpověď:} Takto lze ukončit prac poměr jak zaměstnanec tak i zaměstnavatel.
\subparagraph{Zaměstnanec} nemusí udat důvod výpovědní lhůta je dvouměsíční.
\subparagraph{Zaměstnavatel} musí být uveden důvod (lze se u soudu hájit), v případě že se ruší dané místo (výpověď pro nadbytečnost) je zaměstnanec povinen zaplatit odstupné v min výši 3 měsíčních mezd.

\paragraph{Okamžité zrušení} člověk je vyhozen na základě hrubého porušení prac kázně\footnote{alkohol na pracovišti, krádež na pracovišti}. V případě nepodmíněného odsouzení za úmyslný trestný čin minimálně na jeden rok lze rozvázat prac poměr.

\paragraph{Dohoda} obě strany se dohodnou ns ukončení prac. poměru
\section{Pracovní podmínky mladistvých}
Mladiství smějí vykonávat pouze práce odpovídající jejich fyzickému a psychickému stavu, tzn. nesmí například chodit na noční směny či vykonávat práce přesčas. Mladiství musí mít \textbf{Lékeřskou prohlídku} a absolovat ji minimálně jednou za rok. 

\part{Filozofie}

\chapter{Úvod do filozofie}

Název odvozen od \textit{filo}--miluji, \textit{sofie}--moudrost. Vzniká na východě (v Číně a Indii). Řecká filozofie je vnímána jako první ucelený systém poznatků.

První filozofovése zabívají s problémy vzniku světa. Dnešní filozofové se zajímají o konkrétní problémy člověka.


Aristotelés přichází s rozdělením filozofie na:
\begin{description}
\item[Teoretická]--matematika, fyzika
\item[praktická]--etika, politika
\item[poetická]
\end{description}

V 19. století se vyčleňují ostatní spol. vědy.
Dnes bývá filizofie rozdělována jinak.
\begin{description}
\item[teoretická f.] --

\begin{description}
	\item[Ontologie](\emph{On} -- bytí), věda o bytí a podstatě života(Patří sem dva základní pojmy: \textbf{Materialismus a Idealismus}, Monismus, Dualismus, Pluralismus)
	\item[Gnozeologie](též\emph{Noetika, Epistemiologie}) -- teorie poznání (\emph{Agnoticizmus, Racionalizmus, Senzualizmus, Empirizmus}), patří sem také \textbf{logika}
	
\end{description}
\item[Praktická f.(\textit{aplikovaná})] --
	\begin{description}
	\item[Axiologie] -- (\emph{axia} -- hodnota) věda o hodnotách
	\item[Etika] -- (\emph{éthos} -- zvyk, obyčej) věda o morálce a chování
	\item[Estetika] -- nauka o krásnu
	\item[Filozofická antropologie] -- uvažuje o existenci člověka
	\end{description}
\item[mezní disciplíny] -- 
	\begin{description}
	\item[Filozofie práva]
	\item $\vdots$
	\end{description}
\end{description}

\section{Filozofické pojmy}
\begin{multicols}{2}
\begin{description}
\item[Materialismus],\emph{Materia} -- látka ,tvrdí že základem světa je hmota či látka. (Například atomisté)
\item[Idealismus], \emph{Idea} -- myšlenka (Platón\dots)
\item[Monismus] existuje jedna látka
\item[Dualismus] existují dvě látky
\item[Pluralismus] existuje více látek (např atomisté)
\item[Agnosticizmus] popírá poznatelnost světa
\item[Racionalismus] svět je poznatelný díky rozumu
\item[Senzualismus] smyslové poznání světa
\item[Empirismus] svět je poznatelný díky zkušenostem
\item[logika] věda, která se zabývá o zákony a formy správného usuzování 
\item[Panteismus] existuje jeden bůh, který je všude kolem nás a je za vším.
\end{description}
\end{multicols}

\chapter{Řecká filozofie}

První filozofovése zabívají s problémy vzniku světa. Tito lidé se zajímali mimo jiné častokrát i o jiné vědy.



\section{Milétská škola}
7. století před naším letopočtem

Město Milét bylo jednou z 12 Ionských měst (dnešní Turecko). Milétská škola lze označit za \textbf{školu najivního materializmu}.
\paragraph{Thálés z Milétu (624--548)}
Obchodník, který využíval svých cest k poznání např. Asie či Afriky. Proslavil se v geommetrii a astronomii (předpověděl zatmění slunce).

Tvrdí, že základem světa je voda. Země má podle něj podobu plochého kotouče, který se vznáší na vodě.

\paragraph{Anaximédés} Tvrdí, že základem světa je vzduch. Ten může buď zhoustnout či se může naředit.Zahušťováním vzniká voda, ta v půdu, ta v kámen. Pokud vzduch řídne, přeměňuje se v oheň. Země je podle něj kotouč, který se vznáší ve vzduchu.

Správně stanovil pořadí měsíce a slunce vůči zemi.

\paragraph{Anaximandros} Zbytky díla o přírodě. Sestavil nebeský glóbus kde poskládal postavení planet. Země má podle něj podobu válce obklopenou kulovým nebem. Mícháním látek vzniká \emph{apeiron}, pralátka světa.




\section{Pythágorejci}

Zastupovali idealizmus. Kritika vlády -- \textsf{neexistuje spravedlnost}.

\subsection{Pythágoras ze Samu(580--500)} Byl bohatým obchodníkem. Kritizovali techdejší vládu, tvrdí že neexistuje spravedlnost.

Všechna tělesa se pohybují a přitom vydávají hudbu, kterou neslyšíme (\emph{hudba sfér}).

Obohatil filizofii o slovo \emph{kosmos}

\textbf{Podstatou světa je číslo.} Číslo \textbf{jedna} je součástí všech ostatních -- \textbf{základní číslo}. Číslo 10 je nejlepší ze všech, soušet 1+2+3+4 (1 -- bod, 2 -- přímka, 3 -- rovina, 4 -- prostor).

\subsection{Ostatní pythágorejci}
\paragraph{Filoláos} Říká, že pro život člověka je nejdůležitější \textbf{harmonie} a ta vzniká \textbf{spojováním protikladů}. Příkladem harmonie je sudé a liché, jedno a mnohé, omezené a neomezené, mužské a ženské, dobro a zlo,…

\section{Herakleitos}

Má blízko k materialistům, protože existovala \emph{praenergie}, ta se přeměnila v oheň.

\textbf{Dialektika} -- věda o vývoji. \emph{Oheň může pomalu vychasínat až vyhasne, ale pokud přidáme palivo může hořet dál.}

\textbf{Vše plyne} -- \emph{panthe rei}, aneb dvakrát nevstoupíš do téže řeky.

\section[Elejská škola]{Eleaté}

Je tipické, že na základě logických argumentů vyvracejí pohyb.

\paragraph{Xenofánes z Kolofonu} Kritizoval Homéra za to že dával bohům špatné lidské vlastnosti.

\paragraph{Parmenidés} Říká:

\begin{quote}
\emph{Svět je koule, která je plná. Protože pohyb může existovat jen v prázdnu tak pohyb neexistuje.}
\end{quote}

\paragraph{Zénon z Eleje} Používá slovo dialektika, tvrdí že je to \uv{umění vést logický rozhovor}. Používá \emph{aporie}, důkazy pro vyvrácení pohybu.
\begin{itemize}
\item Letící šíp -- odtud vyplývá, že \textbf{letící šip stojí}.
\begin{quote}
\dots To vyplývá z předpokladu, že čas se skládá ze samých ‚nyní‘. Pokud totiž tento předpoklad neplatí, nevyplyne ani výše zmíněný závěr.
\end{quote}
\item Achylés a želva, Achylés nemůže želvu dohnat, protože ona je vždy o kousek napřed.

\item Dichotomie -- závodník nikdy nemůže do cíle dorazit protože v konečném čase nelze projít nekonečné množství bodů.
\end{itemize}

\section{Mladší fyzikové}

Materialisté.

\paragraph{Empedokles}

Vzdělaný člověk s širokým zájmem (medicína, právo, \dots), věří v reinkarnaci.

Spojuje učení svých předchůdců. Tvrdí že základem světa jsou 4 živly: oheň, voda, země vzduch. 6ivly se pohybují, pohyb způsobují 2 protikladné síly, láska a svár.

\paragraph{Anaxágoras}
Země je plochá, má tvar bubnu. Základem světa jsou semena, kterých je nekonečné množství. Každé semeno má v sobě  zárodek svého opaku. Semena jsou jsou zárodkem nús -- světového rozumu.

\section{Atomisté}
Materialisté. Snažlili racionlální vysvětlení. \emph{Atomos} -- dále nedělitelný, hmota se nedá dělit do nekonečna.

\paragraph{Leukippos}
Představitel kauzalita (příčinnosti) -- všechny děje mají příčinu.

\subsection{Démokritos} Pocházel z rodiny obchodníka. Autor teorie, že základem světa jsou atomy. Atomy se liší \textbf{tvarem,
 velikostí,
 polohou}.
Lidská duše je složena z atomů, ty jsou kulaté. Mluví o tom že atomy se pohybují mechanickým vířivým pohybem, protože každý atom má na svém povrchu háčky. Při srážce se mohou pomocí háčků spojit, či se spojení může rozbýt. Atomy mají primární vlastnosti (hustota, hmotnost, tvrdost), sekundární vlastnosti (chuť, vůně, teplo, \dots). Sekundární vlastnosti vznikly tak, že se lidé domluvili, jaké vlastnosti jimi jsou.

Klade velký důraz a \textbf{etiku}. Morální vlstnosti podle něj stojí nad ostatními.

\section{Sofisté}

Idealistická škola.

 \paragraph{Prothagoras} přichází s myšlenkou že společnost nepotřebuje pravidla, ale stačí se řídit \textbf{právem silnějšího} (těm je dovoleno vše).
 
\section{Sókrates}
Narozen v Athénách, otec byl sochař matka porodní bába. Sokrates se sám vyučil sochařem, avšak nikdy nesochyřil. Oženil se s ž \textsc{Xantipou}. Kvůli jeho rozhovorů byl obžalován ze špatného vlivu na mládež a bezbožství. Byl odsouzen k trestu smrti.\footnote{Sokratés řekl že nelituje toho co dělal a že by to udělal znovu.} 

Nikdy nenapsal žádné dílo, známe ho z díla \emph{Obrana Sokratova} od jeho žáka Platóna.
\begin{large}
\begin{center}

\emph{Vím, že nic nevím.}
\end{center}
\end{large}

Jeho cílem bylo aby se lidé vzdělávali a zajímali o dění.

\paragraph{Sokratovský rozhovor} Velkou část dne trávil na tržištích ve společnosti žáků a příznivců, kde jim kladl různé otázky. Otázky stupňoval do situací kdy nikdo nemohl odpovědět. 
 
 \begin{quote}
 \emph{
Tak jak porodní bába pomáhá na svět, tak by měl učitel pomáhat při vzdělávání. Žák by se měl uřit sokratovským přístupem.
}\end{quote}
 
 \section{Platón}
 
 Žák Sokrata. Všechna jeho díla, krom obrany sokratovy jsou psána formou dialogu. Po Sokratově smrti oppouští Athény, cestuje. Později se vrací zpět a zakládá fil. školu Academia. Je vnímán jako objektivní Idealista či Státovědec.
 
 \paragraph{\emph{Politea} (Ústava):} Dílo se skládá ze dvou částí. \begin{enumerate}
 \item \uv{Ústav je velké množství, vždy je tvoří lidé.}, Kritizuje formy vlády \begin{itemize}
 	\item oligarchie -- Vládnou bohatí, ti ne vždy musí být nejlepšími vládci. Vznikají dva státy, stát bohatých a stát chudých. Ten kdo vlastním přičiněním nemá právo žít mezi ostatními.
 	\item Demokracie -- nejhorší typ vlády. Je vnímána se svobodou, pokud si někdo může dělat co chce tak je to špatně.
 	\item Tyranie -- musí přijí po demokracii, protože jedině tyran může společnost uvést do pořádku. 
	\end{itemize}  

\item Vlastní představa státu. Všechny děti jsou nejprve vzděláváni ve všech dovednostech. Ve dvaceti letech probíhá první \uv{sýto}, ti nejslabší se zachytí ve stavu výrobců. Dále probíhá 10 let vzdělávání, itelektuální školení. Rozdělení společnosti do tří stavů, každý se může dostat do každého stavu\begin{itemize}
\item vládců (vzdělaní, \textbf{moudří} lidé)
\item strážci (odvážní, stateční bojovníci, hlavně velmi dobře ovládají své řemeslo) tomuto stavu je věknováno nejvíce pozornosti. Strážci nesmějí mít žádný majetek. Strážci nemají znát své děti.
\item výrobci (umírnění lidé)
\end{itemize}

Tato vize státu se nikdy nezrealizovala.

 \end{enumerate}

\paragraph{Objektivní idealizmus} Platón tvrdí, že existují dva světy, \textbf{pozemský svět} (ten je nedokonalý) a \textbf{svět idejí} (dokonalý). Na příkladu \emph{Platónské jeskyně} demonstruje že lidé žijí v nedokonalém světě \uv{uvnitř jeskyně} a snažíme se napodobit venkovní svět, svět idejí. Pokud člověk zemře, tak duše opustí tělo a dostává se do světa idejí.
 
 Díla : Kritón  Zákony, Politikos.
  
 \section{Aristoteles}

Žák Platónův, ve 20 nastoupil do jeho Akademie. Otec byl lékař proto měl blízko k přírodním vědám.

Byl učitelem Alexandra Makedonského, se kterým se později spřátelil, díky němu mohl dělat svůj výzkum.

\paragraph{Logika:}(v tehdejší době ji Aristotelés označoval jako \emph{Analytika}). Aristotelés tvrdil že není důležité co si člověk myslí, nýbrž jak myslí. Základem správného myšlení jsou pojmy a aby bylo myšlení správné musí být pojmy správné. Pojem je správný lze-li ho definovat.

Podle něj součástí definice musí být že definovaná věc je součástí nadřazené množiny, ale musí mít speciální vlastnosti. Podle toho tvrdí že jsou pojmy s vyšší obecností a pojmy s nižší obecností.

\begin{quote}
\textbf{Pojem je}, \textsf{pojmenování množiny pojmu nadřazené}, výčet spec. vlastností. 
\end{quote}


Tvrdí že existuje 10 kategorií, které nad sebou nemají pojem vyšší obecnosti. (Substance, Kvalita, Kvantita, Relace). Spojováním pojmů vznikají soudy, ty mohou být pravdivé či nepravdivé. Spojováním soudů vzniká úsudek.

\begin{center}
Pojem + Pojem = Soud (\texttt{True}/\texttt{False})
Soud + soud = úsudek
\end{center}


\begin{quote}
Sokratés je člověk. Člověk je smrtelný. $\Rightarrow$ Sókratés je smrtelný.
\end{quote}

\paragraph{Metafyzika}
doslova \emph{za něčím}. Otázky původu člověka, důvodu jeho života a o původu věcí. Formuloval 4 příčiny jsoucna 
\begin{description}
\item[Látka] materiál, z něhož jsou vyrobeny věci (\emph{causa materialis})
\item[Forma] tvar věcí (\emph{causa formalis}).
\item[Působení]\footnote{(věcí na okolí) někdu tu věc vyrobil} interakce věcí s okolím
\item[Účel] 
\end{description}

Např u člověka je látkou tělo a formou duše.

Duše mohou být u živých věcí  \begin{itemize}
\item vegetativní (rostliny)
\item smyslová (živočichové)
\item rozumová (lidé)
\end{itemize}
Kategorizoval organizmy.

\paragraph{Politika} V tomto díle tvrdí že člověk je \emph{zoon politikon} (tvor společenský), tzn. že k životu potřebuje společnost lidí, lidé zakládají rodiny, z těch jsou obce z těch je stát, ten potřebuje \textbf{vládu, ústavu, obranu}. Společnost podle něj potřebuje pravidla a zákony.

Vláda podle něj může být špatná nebo dobrá.
\begin{center}
\begin{tabular}{p{0.33\textwidth} p{0.33\textwidth}}
Dobré vlády &špatné vlády\\
\textbf{Monarchie} -- vládne jeden & \textbf{Tyranie} -- vláda jednoho špatného\\
\textbf{Aristokracie} --\dotfill & \textbf{Oligarchie} -- \dotfill\\
\textbf{Politea} -- vládnou všichni& \textbf{Demokracie} -- nejhorší typ vlády
\end{tabular}
\end{center}
\section{Stoikéové}

Stoa - sloup, Poikilé

Dělí vědu na \begin{itemize}

\item logiku (dialektika - ve symslu eleatů, rétorika)

\item fyzika -- materialisté, monoteisté

\item etika -- nejdůležitější část fil. systému. Nejdůležitější pro člověka je, aby byl rozumný. Rozumný člověk je schopen rozlišit co je dobro a zlo. Jediným příkladem dobra je cnost a jejím protějškem je nectnost, zbytek není ani dobré ani špatné, nýbrž lhostejné. Pro rozlišení dobra a zla je zapotřebí odbourat afekty, ty ovlivňují naše jednání. Úkolem člověka je boj proti afektům.

\end{itemize}

\chapter{Středověká Filozofie}

\section{Scholastika}
\paragraph{Ranná shcolastika} je charakteristická sporem o \emph{univerzalia}. Zde jsou dva přístupy k obecným pojmům. 

\begin{description}
  \item[Nominalisté] ti tvrdí že obecné pojmy neexistují 
  \item[Realisté] pro ně obecné pojmy existují. 
\end{description}

Již ve středověku se objevují filozofové, kteří označují tento spor za zbytečný např \textsf{Pierre Abelard.}

\section{Arabská filozofie}

Arabský svět se začíná zajímat o Antické učení, zejm. o Aristotela. 

\subparagraph{Al Farabí} Pobýval dlouho v Damašku. Napsal \emph{Traktáty Čistých bratří}, souvisí s jeho členství ve stejnojmenném spolku. 

\paragraph{Východní arab filozofie }

\subparagraph{Abú Alí Ibn Síná (\emph{Avicenna})} byl lékař a filozof. Snažil se arabům přiblížit Aristotelovo učení, zejm logiku. Autor \emph{Knihy poznání.}

\paragraph{Západní arab. filozofie}

\subparagraph{Muhamad Ibn Rušd (\emph{Averroes})} žil ve španělsku. Podrobně studoval Aristotela.

\part{Ekonomie}

\chapter{Úvod do ekonomie}


\paragraph{Ekonomie} je spol věda jejíž název je odvozen z řeckého \emph{oikos}-- hospodářství, domácnost a \emph{nomos} -- zákon, pravidlo. Zabývá se fin. chování subjektů (lidí, firem či státu).
\subparagraph{Mikroekonomie} -- chování  domácností, firem a států

\subparagraph{Makroekonomie} zkoumá ekonomiku jako celek. Díky jednotlivým ukazatelům lze porovnávat úroveň států. Mezi veličiny patř: \texttt{HDP, míra nezaměstnanosti, inflace}\dots

\subparagraph{Normativní} -- popisuje ideální stav (jak snížit nezaměstnanost)

\subparagraph{pozitivní} -- objektivně hodnotí realitu

\paragraph{Ekonomika} Souhrn ekonomických činností, které probíhají v daném subjektu.
\section{Vztah ekonomie k ostatním vědám}
\paragraph{Psychologie} chování lidí na trhu práce, chování lidí při nakupování

\paragraph{Sociologie} průzkumy na trhu

\paragraph{politologie} politická rozhodnutí ovlivňují ekonomiku

\paragraph{Matematika} umožňuje popsat ekonomiku, používají se statistické metody

\section{Ekonomické systémy}

\begin{itemize}
\item\emph{Co vyrbět?}
\item\emph{Pro koho vyrábět?}
\item\emph{Jak vyrábět?}
\end{itemize}

Podle odpovědí na tyto otázky se rozlišují násl. systémy.

\subsection{Tradiční, zvykový, ekonomický systém}

Vychází z rozhodnutí nějaké kmenové rady, co se bude pěstovat, kolik\dots

S rozvojem nástrojů a nových technologií dochází k dělbě práce a může se vyrobit více než se spotřebuje (to si přisvojuje kmenová rada).

\textbf{Příklad:} centrální afrika, amazonský prales

\subsection{Příkazový systém}

Také centrálně plánovaný. Politická špička, elita, rozhoduje o výrobě.

Například v ČSSR během totality. Komunisté naplánovali také výši mezd.

\textbf{Př.} Kuba, KLDR

\subsection{Tržní ekonomika}

Tržní systém v čisté podobě dnes neexistuje.

Trh je řízen \textbf{nabídkou a poptávkou}. 

\subsection{Smíšený systém}

Do tržního systému je zasahováno popřípadě je podporován státem.

\section{Trh}
\textsf{Místo, kde se setkává nabídka s poptávkou.}

Trh je řízen vztahem N-P. Prodávající chce co největší zisk, nakupující chce co nejlevněji nakoupit.

\subparagraph{Trh zboží a služeb} tento trh se odvíjí od ceny.

\subparagraph{Trh práce} zde je ústřední pojmem mzda.

\subparagraph{Trh financí}: půjčování peněz

Pdle velikosti se trhy dělí na 
\begin{description}
\item[Malé] obce, či města
\item[Národní] v rámci státu
\item[Světový] propojeý mezi státy
\item[Dílčí] směna jednoho druho zboží
\item[Agregátní] vyskytují se všechny druhy zboží
\end{description}

\section{Živnost}

\emph{Zákon o živnostenském podnikání 455/1991.}

Živnost je soustavná podnikatelská činnost, provozovaná za účelem dosažení zisku,  vlastním jménem a na vlastní zodpovědnost.

Může ji provozovat každá FO i PO, která získala živnostenské oprávnění. Pro zisk oprávnění je zapotřebí splnit násl. podmínky
\begin{multicols}{2}
\begin{itemize}
\item osoba je svéprávná\footnote{dovršení 18 let či \emph{přivolení soudu} podnikání od 16 let se souhlasem zák zástupce.}
\item Bezúhonost\footnote{Osoba nesmí být odsouzena za úmyslný TČ související s podnikáním.} 
\end{itemize}
\end{multicols}

Živnost nemůže provozovat ten kdo je v konkurzu či mu byl zavrhnut insolovenční návrh.

Druhy živnosti jsou 
\begin{description}
\item[Ohlašovací] po splnění podmínek stačí nahlásit živnost.
\begin{itemize}
\item Řemeslné -- je zapotřebí osvědčení o získání vzdělání \footnote{Výuční list, Maturitní zkouška, Absolovování rekvalifikačního kurzu, Doložených 6 let praxe.}
\item Volné -- není zapotřebí odborná způsobilost, např. \textsf{Realitní činnost, reklamní činnost, \dots}
\item Vázané -- je spojena s odbornou způsobilostí, podnikání v oblasti nakládání s nebezpečnými odpady, oční optika, \dots
\end{itemize}
\item[Koncesované] tyto mohou být porvozovány pouze na základě koncese - \emph{státní povolení pro provozování činnosti}. Stát chce mít kontrolu, dko tyto živnosti vykonnává.

Příklady: výroba, výzkum a prodej výbušnin, činnost spojená se zbraněmi a střelivem, pálenice na pálení alkoholu, 
\end{description}

\section{Obchodní společnosti}

Obchodní společnost je právnická sosoba, většinou založaná více osobami. Vzniká zápisem do obchodního rejstříku.

\paragraph{Veřejná obchodní společnost (v.o.s.)} podpisy zakladatelů musí být úředně ověřeny. za závazky ručí všichni stejným dílem. Zaniká smrtí jednoho ze společníků.

\paragraph{Komanditní společnost} Je tvořena komanditisty (ručí částkou minimálně 5 000 Kč) a komplementáři (ten má vyšší postavení, dává do společnosti totiž vyšší vklad, ručí celým svým majetkem, může vést jednání za celou společnost).

\end{document}
